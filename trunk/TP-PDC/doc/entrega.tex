\documentclass[a4paper,10pt]{article}
\usepackage[utf8x]{inputenc}
\usepackage[spanish]{babel}
\usepackage[utf8x]{inputenc}
\usepackage{verbatim}
\usepackage{hyperref}

\usepackage{anysize}
\marginsize{1.5cm}{1.5cm}{1cm}{1cm}

%\usepackage{sectsty}
%\sectionfont{\Large}

\pdfinfo{%
  /Title    (Trabajo Práctico Especial)
  /Author   ()
  /Creator  ()
  /Producer ()
  /Subject  ()
  /Keywords ()
}

\begin{document}

\begin{titlepage}
\begin{center}
 \huge \underline{\textbf{Trabajo Práctico Especial}}\\[0.05cm]
 \normalsize \textbf{Protocolos de Comunicación}\\[0.25cm]
 \normalsize \textbf{Revisión correspondiente a la entrega: }\\[1cm]

\large
\begin{tabular}{c @{ - } l}
 Andrés Mata Suarez & 50143 \\
 Jimena Pose & 49015 \\
 Pablo Ballesty & 49359 \\
\end{tabular}\\[19.4cm]
 2011 - Segundo cuatrimestre
\end{center}

\end{titlepage}

\setcounter{tocdepth}{2}
\tableofcontents


\newpage

\section{Descripción detallada del protocolo desarrollado}
A continuación se presenta el RFC del protocolo \textit{configurotocol 1.0}, diseñado para manejar la configuración
del servidor proxy.

\verbatiminput{configurotocol2.txt}

\section{Problemas encontrados durante el diseño y la implementación}

\section{Limitaciones de la aplicación}

\section{Posibles extensiones}

\section{Conclusiones}

\section{Ejemplos de testeo}
Para la ejecución de los casos de prueba, se va a contar con los siguientes usuarios:
\begin{itemize}
 \item \textbf{foo} con contraseña \textbf{123123123}
 \item \textbf{bar} con contraseña \textbf{123123123}
 \item \textbf{baz} con contraseña \textbf{123123123}
 \item \textbf{foobar} con contraseña \textbf{123123123}
\end{itemize}

\subsection{Testeos de requerimientos funcionales}

\begin{center}
  \begin{tabular}{|r|p{12.5cm}|}
    \hline
    \textbf{Funcionalidad}	&	Por rango de horarios\\
    \hline
    \textbf{Tipo de test}	&	Positivo\\
    \hline
    \textbf{Descripción}	&	Utilizar el protocolo \textit{configurotocol} para exigir que el usuario
					\textbf{foo} sólo pueda acceder de 6:00 a 18:00 horas. Iniciar sesión con
					dicho usuario en dicho rango horario.\\
    \hline
    \textbf{Resultado esperado}	&	El usuario \textbf{foo} inicia sesión sin problemas.\\
    \hline   
  \end{tabular}
\end{center}

\begin{center}
  \begin{tabular}{|r|p{12.5cm}|}
    \hline
    \textbf{Funcionalidad}	&	Por rango de horarios\\
    \hline
    \textbf{Tipo de test}	&	Negativo\\
    \hline
    \textbf{Descripción}	&	Utilizar el protocolo \textit{configurotocol} para exigir que el usuario
					\textbf{foo} sólo pueda acceder de 6:00 a 18:00 horas. Iniciar sesión con
					dicho usuario fuera de ese rango horario.\\
    \hline
    \textbf{Resultado esperado}	&	El usuario \textbf{foo} no tiene permitido el inicio de sesión en dicho
					rango horario. Se recibe un error acorde al protocolo.\\
    \hline   
  \end{tabular}
\end{center}

\begin{center}
  \begin{tabular}{|r|p{12.5cm}|}
    \hline
    \textbf{Funcionalidad}	&	Por rango de horarios\\
    \hline
    \textbf{Tipo de test}	&	Negativo\\
    \hline
    \textbf{Descripción}	&	Utilizar el protocolo \textit{configurotocol} para exigir que el usuario
					\textbf{foo} sólo pueda acceder de 19:00 a 06:00 horas. Iniciar sesión con
					dicho usuario dentro de ese rango horario.\\
    \hline
    \textbf{Resultado esperado}	&	El usuario \textbf{foo} no tiene permitido el inicio de sesión en dicho
					rango horario. Se recibe un error acorde al protocolo. Que el configurotocol permita este seteo.\\
    \hline   
  \end{tabular}
\end{center}

\begin{center}
  \begin{tabular}{|r|p{12.5cm}|}
    \hline
    \textbf{Funcionalidad}	&	Por cantidad de logins exitosos por usuario y día\\
    \hline
    \textbf{Tipo de test}	&	Positivo\\
    \hline
    \textbf{Descripción}	&	Asegurarse de que el usuario \textbf{bar} no haya iniciado sesión
					anteriormente en el día.
					Utilizar el protocolo \textit{configurotocol} para exigir que el usuario
					\textbf{bar} sólo pueda acceder al sistema un máximo de 1 (una) vez
					por día. Iniciar sesión en el sistema como dicho usuario.\\
    \hline
    \textbf{Resultado esperado}	&	El usuario \textbf{bar} inicia sesión sin problemas.\\
    \hline   
  \end{tabular}
\end{center}

\begin{center}
  \begin{tabular}{|r|p{12.5cm}|}
    \hline
    \textbf{Funcionalidad}	&	Por cantidad de logins exitosos por usuario y día\\
    \hline
    \textbf{Tipo de test}	&	Negativo\\
    \hline
    \textbf{Descripción}	&	Asegurarse de que el usuario \textbf{bar} no haya iniciado sesión
					anteriormente en el día.
					Utilizar el protocolo \textit{configurotocol} para exigir que el usuario
					\textbf{bar} sólo pueda acceder al sistema un máximo de 1 (una) vez
					por día. Iniciar sesión en el sistema como dicho usuario. Cerrar sesión.
					Iniciar sesión una vez más.\\
    \hline
    \textbf{Resultado esperado}	&	El usuario \textbf{bar} ya cumplió su cuota diaria de accesos.
					El segundo login no es aceptado. Se devuelve mensaje de error acorde al protocolo.\\
    \hline   
  \end{tabular}
\end{center}

\begin{center}
  \begin{tabular}{|r|p{12.5cm}|}
    \hline
    \textbf{Funcionalidad}	&	Por lista negra (dirección IP)\\
    \hline
    \textbf{Tipo de test}	&	Positivo\\
    \hline
    \textbf{Descripción}	&	Utilizar el protocolo \textit{configurotocol} para impedir conexiones
					entrantes de la dirección 192.168.1.50. Iniciar sesión como
					\textbf{baz} desde la dirección 192.168.1.51.\\
    \hline
    \textbf{Resultado esperado}	&	El usuario \textbf{baz} inicia sesión sin problemas.\\
    \hline   
  \end{tabular}
\end{center}

\begin{center}
  \begin{tabular}{|r|p{12.5cm}|}
    \hline
    \textbf{Funcionalidad}	&	Por lista negra (dirección IP)\\
    \hline
    \textbf{Tipo de test}	&	Negativo\\
    \hline
    \textbf{Descripción}	&	Utilizar el protocolo \textit{configurotocol} para impedir conexiones
					entrantes de la dirección 192.168.1.50. Iniciar sesión como
					\textbf{baz} desde la dirección 192.168.1.50.\\
    \hline
    \textbf{Resultado esperado}	&	La dirección 192.168.1.50 se encuentra en la lista negra. No se permite
					el inicio de sesión. Se devuelve mensaje de error acorde al protocolo.\\
    \hline   
  \end{tabular}
\end{center}

\begin{center}
  \begin{tabular}{|r|p{12.5cm}|}
    \hline
    \textbf{Funcionalidad}	&	Por lista negra (redes IP)\\
    \hline
    \textbf{Tipo de test}	&	Positivo\\
    \hline
    \textbf{Descripción}	&	Utilizar el protocolo \textit{configurotocol} para impedir conexiones
					entrantes del rango de direcciones 192.168.1.0/25. Iniciar sesión como
					\textbf{foobar} desde la dirección 192.168.1.130. Iniciar sesión como
					\textbf{baz} desde la dirección 192.168.1.254.\\
    \hline
    \textbf{Resultado esperado}	&	Ambos usuarios inician sesión sin problemas.\\
    \hline   
  \end{tabular}
\end{center}

\begin{center}
  \begin{tabular}{|r|p{12.5cm}|}
    \hline
    \textbf{Funcionalidad}	&	Por lista negra (redes IP)\\
    \hline
    \textbf{Tipo de test}	&	Negativo\\
    \hline
    \textbf{Descripción}	&	Utilizar el protocolo \textit{configurotocol} para impedir conexiones
					entrantes del rango de direcciones 192.168.1.0/25. Iniciar sesión como
					\textbf{foobar} desde la dirección 192.168.1.126. Iniciar sesión como
					\textbf{bar} desde la dirección 192.168.1.10.\\
    \hline
    \textbf{Resultado esperado}	&	No se permite ninguno de los dos inicios de sesión. Se devuelven mensajes
					acordes para cada instancia de la aplicación.\\
    \hline   
  \end{tabular}
\end{center}

\begin{center}
  \begin{tabular}{|r|p{12.5cm}|}
    \hline
    \textbf{Funcionalidad}	&	Por cantidad de sesiones concurrentes\\
    \hline
    \textbf{Tipo de test}	&	Positivo\\
    \hline
    \textbf{Descripción}	&	Utilizar el protocolo \textit{configurotocol} para restringir la cantidad
					máxima de sesiones concurrentes del usuario \textbf{baz} a 3.
					Iniciar sesión como dicho usuario desde 2 clientes distintos.\\
    \hline
    \textbf{Resultado esperado}	&	Se permite el inicio de sesión en cada instancia del programa.\\
    \hline   
  \end{tabular}
\end{center}

\begin{center}
  \begin{tabular}{|r|p{12.5cm}|}
    \hline
    \textbf{Funcionalidad}	&	Por cantidad de sesiones concurrentes\\
    \hline
    \textbf{Tipo de test}	&	Positivo\\
    \hline
    \textbf{Descripción}	&	Utilizar el protocolo \textit{configurotocol} para restringir la cantidad
					máxima de sesiones concurrentes del usuario \textbf{baz} a 3.
					Iniciar sesión como dicho usuario desde 3 clientes
					distintos. Utilizar un nuevo cliente e iniciar sesión nuevamente.\\
    \hline
    \textbf{Resultado esperado}	&	Se permite el inicio de sesión en el último cliente, y el primer cliente
					que se utilizó se desconecta recibiendo un error acorde al protocolo.\\
    \hline   
  \end{tabular}
\end{center}

\begin{center}
  \begin{tabular}{|r|p{12.5cm}|}
    \hline
    \textbf{Funcionalidad}	&	Por silenciamiento de usuarios\\
    \hline
    \textbf{Tipo de test}	&	Positivo\\
    \hline
    \textbf{Descripción}	&	Utilizar el protocolo \textit{configurotocol} para asegurarse de que el usuario
					\textbf{foobar} no se encuentre silenciado. Iniciar sesión como dicho
					usuario y emitir mensajes hacia el usuario \textbf{foo} que estará conectado.
					Luego el usuario \textbf{foo} emite mensajes hacia \textbf{foobar}.\\
    \hline
    \textbf{Resultado esperado}	&	Los usuarios se comunican sin problemas.\\
    \hline   
  \end{tabular}
\end{center}

\begin{center}
  \begin{tabular}{|r|p{12.5cm}|}
    \hline
    \textbf{Funcionalidad}	&	Por silenciamiento de usuarios\\
    \hline
    \textbf{Tipo de test}	&	Negativo\\
    \hline
    \textbf{Descripción}	&	Utilizar el protocolo \textit{configurotocol} para silenciar al usuario
					\textbf{foobar}. Iniciar sesión como dicho usuario y emitir mensajes hacia
					el usuario \textbf{foo} que estará conectado.
					Luego \textbf{foo} envía mensajes hacia \textbf{foobar} y hacia \textbf{bar}.\\
    \hline
    \textbf{Resultado esperado}	&	\textbf{foo} y \textbf{foobar} no pueden comunicarse, y \textbf{bar} recibe los mensajes de \textbf{foo}.\\
    \hline   
  \end{tabular}
\end{center}

\begin{center}
  \begin{tabular}{|r|p{12.5cm}|}
    \hline
    \textbf{Funcionalidad}	&	Filtros(L33t)\\
    \hline
    \textbf{Tipo de test}	&	Positivo\\
    \hline
    \textbf{Descripción}	&	Mediante \textit{configurotocol} activar el filtro de \textbf{L33t}. 
					Emitir mensajes desde \textbf{foo} hacia \textbf{bar} que posean vocales.\\
    \hline
    \textbf{Resultado esperado}	&	\textbf{bar} recibe los mensajes en formato \textbf{L33t}.\\
    \hline   
  \end{tabular}
\end{center}

\begin{center}
  \begin{tabular}{|r|p{12.5cm}|}
    \hline
    \textbf{Funcionalidad}	&	Filtros(L33t)\\
    \hline
    \textbf{Tipo de test}	&	Positivo\\
    \hline
    \textbf{Descripción}	&	Mediante \textit{configurotocol} activar el filtro de \textbf{L33t}. 
					Emitir mensajes desde \textbf{foo} hacia \textbf{bar} que posean vocales.\\
    \hline
    \textbf{Resultado esperado}	&	\textbf{bar} recibe los mensajes en formato \textbf{L33t}.\\
    \hline   
  \end{tabular}
\end{center}

\begin{center}
  \begin{tabular}{|r|p{12.5cm}|}
    \hline
    \textbf{Funcionalidad}	&	Filtros(Hash)\\
    \hline
    \textbf{Tipo de test}	&	Positivo\\
    \hline
    \textbf{Descripción}	&	Mediante \textit{configurotocol} activar el filtro de \textbf{Hash}. 
					Enviar un archivo desde \textbf{foo} hacia \textbf{bar}. Asegurarse que 
					se envía el hash correspondiente.\\
    \hline
    \textbf{Resultado esperado}	&	\textbf{bar} recibe el archivo sin problemas, con el hash generado.\\
    \hline   
  \end{tabular}
\end{center}

\begin{center}
  \begin{tabular}{|r|p{12.5cm}|}
    \hline
    \textbf{Funcionalidad}	&	Filtros(Hash)\\
    \hline
    \textbf{Tipo de test}	&	Positivo\\
    \hline
    \textbf{Descripción}	&	Mediante \textit{configurotocol} activar el filtro de \textbf{Hash}. 
					Enviar un archivo desde \textbf{foo} hacia \textbf{bar}. Quitar el hash del mensaje.\\
    \hline
    \textbf{Resultado esperado}	&	\textbf{bar} recibe el archivo sin problemas, con el hash generado.\\
    \hline   
  \end{tabular}
\end{center}

\begin{center}
  \begin{tabular}{|r|p{12.5cm}|}
    \hline
    \textbf{Funcionalidad}	&	Filtros(Hash)\\
    \hline
    \textbf{Tipo de test}	&	Negativo\\
    \hline
    \textbf{Descripción}	&	Mediante \textit{configurotocol} activar el filtro de \textbf{Hash}. 
					Enviar un archivo desde \textbf{foo} hacia \textbf{bar}. Alterar el hash.\\
    \hline
    \textbf{Resultado esperado}	&	\textbf{bar} no recibe el archivo.\\
    \hline   
  \end{tabular}
\end{center}

\subsection{Testeos de requerimientos no funcionales}

\begin{center}
  \begin{tabular}{|r|p{12.5cm}|}
    \hline
    \textbf{Requerimiento no funcional}	& Envío de archivos\\
    \hline
    \textbf{Tipo de test}	&	Positivo\\
    \hline
    \textbf{Descripción}	&	Desde el usuario \textbf{foo} enviar un archivo a \textbf{bar} de un tamaño de 30Mb.\\
    \hline
    \textbf{Resultado esperado}	&	\textbf{bar} recibe el archivo sin problemas.\\
    \hline   
  \end{tabular}
\end{center}

\begin{center}
  \begin{tabular}{|r|p{12.5cm}|}
    \hline
    \textbf{Requerimiento no funcional}	& Concurrencia\\
    \hline
    \textbf{Tipo de test}	&	Positivo\\
    \hline
    \textbf{Descripción}	&	Configurar JMeter para que realice \textit{n} conexiones al servidor proxy y emita mensajes.
					Ir aumentando \textit{n}.\\
    \hline
    \textbf{Resultado esperado}	&	Para n aceptable (analizaremos durante el desarrollo cuanto sería un n aceptable) no se rechazan conexiones y pueden emitirse los mensajes.\\
    \hline   
  \end{tabular}
\end{center}

\section{Guía de instalación detallada y precisa}

\section{Instrucciones para la configuración}

\section{Ejemplos de configuración}

\section{Documento de diseño del proyecto}

\end{document}
