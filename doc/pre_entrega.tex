\documentclass[a4paper,10pt]{article}
\usepackage[utf8x]{inputenc}
\usepackage[spanish]{babel}
\usepackage[utf8x]{inputenc}
\usepackage{verbatim}

\usepackage{anysize}
\marginsize{1.5cm}{1.5cm}{1cm}{1cm}

%\usepackage{sectsty}
%\sectionfont{\Large}

\pdfinfo{%
  /Title    (Pre-entrega Trabajo Práctico Especial)
  /Author   ()
  /Creator  ()
  /Producer ()
  /Subject  ()
  /Keywords ()
}

\begin{document}

\begin{titlepage}
\begin{center}
 \huge \underline{\textbf{Pre-entrega Trabajo Práctico Especial}}\\[0.05cm]
 \normalsize \textbf{Protocolos de Comunicación}\\[1cm]

\large
\begin{tabular}{c @{ - } l}
 Andrés Mata Suarez & 50143 \\
 Jimena Pose & 49015 \\
 Pablo Ballesty & 49359 \\
\end{tabular}\\[21.4cm]
 2011 - Segundo cuatrimestre
\end{center}

\end{titlepage}

\setcounter{tocdepth}{2}
\tableofcontents


\newpage
\section{Documentación relevante para el desarrollo}

Para el análsis del trabajo práctico se utilizan los siguientes documentos
\begin{itemize}
 \item RFC 6120: Extensible Messaging and Presence Protocol (XMPP): Core
 \item RFC 6121: Extensible Messaging and Presence Protocol (XMPP): Instant Messaging and Presence
 \item RFC 6122: Extensible Messaging and Presence Protocol (XMPP): Address Format

\end{itemize}

\section{Protocolos a desarrollar}
A continuación se presenta el RFC del protocolo \textit{configurotocol 1.0}, diseñado para manejar la configuración
del servidor proxy.

\verbatiminput{configurotocol.txt}

\section{Potenciales problemas y dificultades}
\subsection{Transparencia}
El proxy XMPP a desarrollar debería actuar de manera completamente invisible frente a cada par de entidades (cliente y servidor) conectadas.
En otras palabras, el flujo de información que envía una parte debería arribar sin ningún tipo de alteración al otro extremo
de la conexión, siempre y cuando no estén activas ninguna de las funciones de transformación ofrecidas por la aplicación. Por su parte, el resto
de los requerimientos funcionales no deberían modificar en lo más mínimo el \textit{stream} XML enviado entre pares.

\subsection{Concurrencia}
Frente a cantidades superlativas de conexiones, el proxy debería estar diseñado para trabajar de manera performante, optimizando el manejo de hilos de ejecución.
Para tal fin, se opta por la implementación de un patrón de \textit{pool} de \textit{threads}. La creación atolondrada de \textit{threads} podría consumir demasiados recursos
en un período relativamente corto de tiempo; por el contrario, la constante reutilización de hilos de ejecución para nuevas conexiones podría provocar un verdadero
cuello de botella en la performance y el tiempo de ejecución de la aplicación. El objetivo de este inciso radica en encontrar el balance correcto entre la creación y reutilización de threads,
de tal manera que el sistema funcione en condiciones óptimas.

Por supuesto, la destrucción criteriosa de hilos inactivos es un factor tan importante como sendos anteriores. De más está decir que el consumo de recursos del sistema debería
ser el mínimo posible.

\subsection{Manejo de grandes \textit{streams} de información}
Los \textit{streams} de información que se consideren de gran tamaño deben ser almacenados temporalmente en disco. Obviar esta política implicaría riesgo de agotación de la memoria y, en consecuencia,
la pérdida de la estabilidad deseada en la aplicación. Debido a esto, es necesario implementar un módulo eficiente con la función de mantener ciertos \textit{streams}
en disco e ir llevándolos a memoria a medida que sea necesario. La aplicación no debería dejar de funcionar (o comenzar a funcionar erráticamente) debido al tamaño de los
\textit{streams} que se manejen.

De nuevo, se espera que el manejo de recursos sea eficiente: no es deseable la creación excesiva de archivos en disco así como tampoco lo es la no eliminación de archivos que ya no están en uso
por la aplicación.

\subsection{Controles y aplicaciones externas}
En caso de que algún control efectuado por la aplicación de proxy (por ej.: control de accesos) tenga efecto sobre cierta entidad en una conexión, el mensaje de error devuelto deberá tener
la forma de un mensaje de error XMPP. Una aplicación externa que utilize el proxy no debería tener motivos para sospechar que un error del tipo mencionado no proviene del otro
\textit{endpoint} de conexión.


\section{Ambiente de desarrollo y testing}
\section{Casos de prueba a realizar}
Para cada caso de testeo, se consideran los siguientes usuarios:
\begin{itemize}
  \item \textbf{cthulhu}, contraseña \textbf{fhtagn}.\\
	Usuario con permisos administrativos. Cada vez que se pide utilizar el protocolo \textit{configurotocol}
	para realizar algún tipo de configuración, se asume que se está utilizando a este usuario.
  \item \textbf{dagon}, contraseña \textbf{dagon}.
  \item \textbf{shubniggurath}, contraseña \textbf{shubniggurath}.
\end{itemize}

\subsection{XXXXXXXXXXX CASOS DE TESTEO QUE FALTAN XXXXXXXXXXXXXX}
\textbf{!!!!!ANOTO LOS QUE NO HICE, NO SE SI HAY QUE HACER PARA TODOS IGUALMENTE !!!!!!!!!!!!!}\\
\textbf{!!!!! EN LOS CASOS DE TESTEO CORRESPONDIENTES A BLACKLIST, FALTAN METER VALORES NUMERICOS PARA LAS IP !!!!!!!!!}
\begin{itemize}
 \item Modo de uso
 \item concurrencia
 \item encadenamiento de proxis
 \item logging
 \item multiplexador de cuentas
 \item transofrmaciones
\end{itemize}

\subsection{Controles}

\begin{center}
  \begin{tabular}{|r|p{12.5cm}|}
    \hline
    \textbf{Funcionalidad}	&	Por rango de horarios\\
    \hline
    \textbf{Tipo de test}	&	Positivo\\
    \hline
    \textbf{Descripción}	&	Utilizar el protocolo \textit{configurotocol} para exigir que el usuario
					\textbf{dagon} sólo pueda acceder de 6:00 a 18:00 horas. Iniciar sesión con
					dicho usuario en dicho rango horario.\\
    \hline
    \textbf{Resultado esperado}	&	El usuario \textbf{dagon} inicia sesión sin problemas.\\
    \hline   
  \end{tabular}
\end{center}

\begin{center}
  \begin{tabular}{|r|p{12.5cm}|}
    \hline
    \textbf{Funcionalidad}	&	Por rango de horarios\\
    \hline
    \textbf{Tipo de test}	&	Negativo\\
    \hline
    \textbf{Descripción}	&	Utilizar el protocolo \textit{configurotocol} para exigir que el usuario
					\textbf{dagon} sólo pueda acceder de 6:00 a 18:00 horas. Iniciar sesión con
					dicho usuario fuera de ese rango horario.\\
    \hline
    \textbf{Resultado esperado}	&	El usuario \textbf{dagon} no tiene permitido el inicio de sesión en dicho
					rango horario. Se devuelve mensaje de error acorde.\\
    \hline   
  \end{tabular}
\end{center}

\begin{center}
  \begin{tabular}{|r|p{12.5cm}|}
    \hline
    \textbf{Funcionalidad}	&	Por cantidad de logins exitosos por usuario y día\\
    \hline
    \textbf{Tipo de test}	&	Positivo\\
    \hline
    \textbf{Descripción}	&	Asegurarse de que el usuario \textbf{dagon} no haya iniciado sesión
					anteriormente en el día.
					Utilizar el protocolo \textit{configurotocol} para exigir que el usuario
					\textbf{dagon} sólo pueda acceder al sistema un máximo de 1 (una) vez
					por día. Iniciar sesión en el sistema como dicho usuario.\\
    \hline
    \textbf{Resultado esperado}	&	El usuario \textbf{dagon} inicia sesión sin problemas.\\
    \hline   
  \end{tabular}
\end{center}

\begin{center}
  \begin{tabular}{|r|p{12.5cm}|}
    \hline
    \textbf{Funcionalidad}	&	Por cantidad de logins exitosos por usuario y día\\
    \hline
    \textbf{Tipo de test}	&	Negativo\\
    \hline
    \textbf{Descripción}	&	Asegurarse de que el usuario \textbf{dagon} no haya iniciado sesión
					anteriormente en el día.
					Utilizar el protocolo \textit{configurotocol} para exigir que el usuario
					\textbf{dagon} sólo pueda acceder al sistema un máximo de 1 (una) vez
					por día. Iniciar sesión en el sistema como dicho usuario. Cerrar sesión.
					Iniciar sesión una vez más.\\
    \hline
    \textbf{Resultado esperado}	&	El usuario \textbf{dagon} ya cumplió su cuota diaria de accesos.
					El segundo login no es aceptado. Se devuelve mensaje de error acorde.\\
    \hline   
  \end{tabular}
\end{center}

\begin{center}
  \begin{tabular}{|r|p{12.5cm}|}
    \hline
    \textbf{Funcionalidad}	&	Por lista negra (dirección IP)\\
    \hline
    \textbf{Tipo de test}	&	Positivo\\
    \hline
    \textbf{Descripción}	&	Utilizar el protocolo \textit{configurotocol} para impedir conexiones
					entrantes de la dirección XXX.XXX.XXX.XXX. Iniciar sesión como
					\textbf{dagon} desde la dirección YYY.YYY.YYY.YYY.\\
    \hline
    \textbf{Resultado esperado}	&	El usuario \textbf{dagon} inicia sesión sin problemas.\\
    \hline   
  \end{tabular}
\end{center}

\begin{center}
  \begin{tabular}{|r|p{12.5cm}|}
    \hline
    \textbf{Funcionalidad}	&	Por lista negra (dirección IP)\\
    \hline
    \textbf{Tipo de test}	&	Negativo\\
    \hline
    \textbf{Descripción}	&	Utilizar el protocolo \textit{configurotocol} para impedir conexiones
					entrantes de la dirección XXX.XXX.XXX.XXX. Iniciar sesión como
					\textbf{dagon} desde la dirección XXX.XXX.XXX.XXX.\\
    \hline
    \textbf{Resultado esperado}	&	La dirección XXX.XXX.XXX.XXX se encuentra en la lista negra. No se permite
					el inicio de sesión. Se devuelve mensaje de error acorde.\\
    \hline   
  \end{tabular}
\end{center}

\begin{center}
  \begin{tabular}{|r|p{12.5cm}|}
    \hline
    \textbf{Funcionalidad}	&	Por lista negra (redes IP)\\
    \hline
    \textbf{Tipo de test}	&	Positivo\\
    \hline
    \textbf{Descripción}	&	Utilizar el protocolo \textit{configurotocol} para impedir conexiones
					entrantes del rango de direcciones XXX.XXX.XXX.XXX/ZZ. Iniciar sesión como
					\textbf{dagon} desde la dirección YYY.YYY.YYY.YYY. Iniciar sesión como
					\textbf{shubniggurath} desde la dirección VVV.VVV.VVV.VVV.\\
    \hline
    \textbf{Resultado esperado}	&	Ambos usuarios inician sesión sin problemas.\\
    \hline   
  \end{tabular}
\end{center}

\begin{center}
  \begin{tabular}{|r|p{12.5cm}|}
    \hline
    \textbf{Funcionalidad}	&	Por lista negra (redes IP)\\
    \hline
    \textbf{Tipo de test}	&	Negativo\\
    \hline
    \textbf{Descripción}	&	Utilizar el protocolo \textit{configurotocol} para impedir conexiones
					entrantes del rango de direcciones XXX.XXX.XXX.XXX/ZZ. Iniciar sesión como
					\textbf{dagon} desde la dirección YYY.YYY.YYY.YYY. Iniciar sesión como
					\textbf{shubniggurath} desde la dirección VVV.VVV.VVV.VVV.\\
    \hline
    \textbf{Resultado esperado}	&	No se permite ninguno de los dos inicios de sesión. Se devuelven mensajes
					acordes para cada instancia de la aplicación.\\
    \hline   
  \end{tabular}
\end{center}

\begin{center}
  \begin{tabular}{|r|p{12.5cm}|}
    \hline
    \textbf{Funcionalidad}	&	Por cantidad de sesiones concurrentes\\
    \hline
    \textbf{Tipo de test}	&	Positivo\\
    \hline
    \textbf{Descripción}	&	Utilizar el protocolo \textit{configurotocol} para restringir la cantidad
					máxima de sesiones concurrentes del usuario \textbf{dagon} a 3.
					Iniciar sesión como dicho usuario desde 2 instancias de la aplicación
					distintas.\\
    \hline
    \textbf{Resultado esperado}	&	Se permite el inicio de sesión en cada instancia del programa.\\
    \hline   
  \end{tabular}
\end{center}

\begin{center}
  \begin{tabular}{|r|p{12.5cm}|}
    \hline
    \textbf{Funcionalidad}	&	Por cantidad de sesiones concurrentes\\
    \hline
    \textbf{Tipo de test}	&	Negativo\\
    \hline
    \textbf{Descripción}	&	Utilizar el protocolo \textit{configurotocol} para restringir la cantidad
					máxima de sesiones concurrentes del usuario \textbf{dagon} a 3.
					Iniciar sesión como dicho usuario desde 3 instancias de la aplicación
					distintas. Ejecutar una nueva instancia e iniciar sesión como el mismo
					usuario una vez más.\\
    \hline
    \textbf{Resultado esperado}	&	Se permite el inicio de sesión en todas las instancias excepto en la última.
					Se devuelve mensaje de error acorde en esta última.\\
    \hline   
  \end{tabular}
\end{center}

\begin{center}
  \begin{tabular}{|r|p{12.5cm}|}
    \hline
    \textbf{Funcionalidad}	&	Por silenciamiento de usuarios\\
    \hline
    \textbf{Tipo de test}	&	Positivo\\
    \hline
    \textbf{Descripción}	&	Utilizar el protocolo \textit{configurotocol} para asegurarse de que el usuario
					\textbf{dagon} no se encuentre silenciado. Iniciar sesión como dicho
					usuario y empezar a ejecutar comandos XMPP.\\
    \hline
    \textbf{Resultado esperado}	&	El usuario \textbf{dagon} puede comunicarse sin problemas.\\
    \hline   
  \end{tabular}
\end{center}

\begin{center}
  \begin{tabular}{|r|p{12.5cm}|}
    \hline
    \textbf{Funcionalidad}	&	Por silenciamiento de usuarios\\
    \hline
    \textbf{Tipo de test}	&	Negativo\\
    \hline
    \textbf{Descripción}	&	Utilizar el protocolo \textit{configurotocol} para silenciar al usuario
					\textbf{dagon}. Iniciar sesión como dicho usuario y empezar a ejecutar
					comandos XMPP.\\
    \hline
    \textbf{Resultado esperado}	&	El usuario \textbf{dagon} se encuentra silenciado. Se devuelve mensaje
					de error acorde.\\
    \hline   
  \end{tabular}
\end{center}

\end{document}
